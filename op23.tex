%% Copyright (c) 2023, Tom M. Ragonneau
\documentclass[
    % nolicense,  % Whether to include the license frame
]{talk}

\newcommand*{\ceq}{h}
\newcommand*{\cub}{g}
\newcommand*{\iter}[1][]{x\ifblank{#1}{}{_{#1}}}
\newcommand*{\lag}{\mathcal{L}}
\newcommand*{\lm}[1][]{\lambda\ifblank{#1}{}{_{#1}}}
\newcommand*{\obj}{f}
\newcommand*{\step}[1][]{s\ifblank{#1}{}{_{#1}}}
\newcommand*{\xl}{l}
\newcommand*{\xu}{u}

% Headings
% \title{Some perspectives on \glsfmtshort{sqp}}
% \subtitle{A new manifold interpretation}
\title{COBYQA}
\subtitle{A derivative-free trust-region SQP method for nonlinearly constrained optimization}
\date{OP23 (June 3, 2023)}
\author{\href{https://www.tomragonneau.com}{\textbf{Tom M. Ragonneau}} \and \href{https://www.zhangzk.net}{Zaikun Zhang}}
\institute{
    Department of Applied Mathematics\\
    The Hong Kong Polytechnic University\\
    Hung Hom, Kowloon, Hong Kong
}
\titlegraphic{}
\hypersetup{
    pdfsubject={OP23},
    pdfkeywords={},
}

\begin{document}

\maketitle

\begin{frame}{General context}
    We designed a method named COBYQA for solving
    \begin{equation*}
        \min_{\iter \in \R^n} \quad \obj(\iter) \quad \text{s.t.} \quad
        \begin{cases}
            \cub(\iter) \le 0,\\
            \ceq(\iter) = 0,\\
            \xl \le \iter \le \xu,
        \end{cases}
    \end{equation*}
    where derivatives of~$\obj$,~$\cub$, and~$\ceq$ are \alert{unavailable}.

    \medskip

    \begin{block}{Notes on the method}
        \begin{itemize}
            \item COBYQA aims at being a \alert{successor} to COBYLA.
            \item We \alert{implemented} COBYQA into a Python solver.
            \item The bound constraints are assumed \alert{inviolable}.
        \end{itemize}
    \end{block}
\end{frame}

\begin{frame}{The method}
    \begin{block}{How does it work?}
        \begin{itemize}
            \item COBYQA is a derivative-free trust-region \alert{SQP} method.
            \item It builds \alert{models} of~$\obj$,~$\cub$, and~$\ceq$ by underdetermined interpolation.
            \item Every point visited by COBYQA satisfied the \alert{bound} constraints.
        \end{itemize}
    \end{block}

    \medskip

    We implemented COBYQA in Python and made it \alert{publicly available}.

    \begin{center}
        \begin{center}
            \qrcode{https://www.cobyqa.com}\\[1ex]
            \href{https://www.cobyqa.com}{www.cobyqa.com}
        \end{center}
    \end{center}
\end{frame}

\begin{frame}{The focus of this talk}
    \begin{block}{}
        \begin{itemize}
            \item We provide a new \alert{interpretation} of SQP.
            \item We study some connections with \alert{manifold optimization}.
        \end{itemize}
    \end{block}
\end{frame}

\begin{frame}{Table of contents}
    \tableofcontents[hideallsubsections]
\end{frame}

\section{Brief review of SQP}

\begin{frame}{To do.}
    In this talk, we consider the problem
    \begin{equation*}
        \min_{\iter \in \R^n} \quad \obj(\iter) \quad \text{s.t.} \quad \ceq(\iter) = 0.
    \end{equation*}
    Given an iterate~$\iter[k]$, SQP sets~$\step[k]$ to a solution to
    \begin{equation*}
        \begin{aligned}
            \min_{\step \in \R^n}   & \quad \obj(\iter[k]) + \step^{\T} \nabla \obj(\iter[k]) + \frac{1}{2} \step^{\T} \nabla_{\iter, \iter}^2 \lag(\iter[k], \lm[k]) \step\\
            \text{s.t.}             & \quad \ceq(\iter[k]) + \nabla \ceq(\iter[k]) \step = 0,
        \end{aligned}
    \end{equation*}
    for some Lagrange multiplier~$\lm[k]$, and updates~$\iter[k + 1] = \iter[k] + \step[k]$.
\end{frame}

\section{A new interpretation of SQP}

\begin{frame}{To do.}
\end{frame}

\section{SQP and manifold optimization}

\begin{frame}{To do.}
\end{frame}

\section{Conclusion}

\begin{frame}{Conclusion}
\end{frame}


\appendix
\ifnum\value{cite}>0
    \begin{frame}[t,allowframebreaks]{References}
        \bibliographystyle{apalike}
        \bibliography{\bibfile}
    \end{frame}
\fi

\end{document}
